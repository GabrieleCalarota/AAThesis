\chapter{Introduction}
\section{Motivation and Problem Statement}
The task of determining or verifying the authorship of an anonymous text based solely on internal evidence is a very old one, dating back at least to the medieval scholastics, for whom the reliable attribution of a given text to a known ancient authority was essential to determining the text’s veracity. More recently, this problem of authorship attribution has gained greater prominence due to new applications in forensic analysis, humanities scholarship, and electronic commerce, and the development of computational methods for addressing the problem.
In the simplest form of the problem, we are given examples
of the writing of a number of candidate authors and are asked
to determine which of them authored a given anonymous
ext. In this straightforward form, the authorship attribution
problem fits the standard modern paradigm of a text categorization problem. The components of text categorization systems are by
now fairly well understood: Documents are represented as
numerical vectors that capture statistics of potentially relevant features of the text, and machine learning methods are
used to find classifiers that separate documents that belong
to different classes.
A scientific approach to the authorship attribution problem was first proposed in the late 19th century in the work of Mendenhall (1887), who studied the authorship of texts attributed to Bacon, Marlowe, and Shakespeare.
The key idea was that
the writing of each author could be characterized by a unique
curve expressing the relationship between word length and
relative frequency of occurrence; these characteristic curves
thus would provide a basis for author attribution of anonymous texts. This early work was put on a firmer statistical
basis in the early 20th century with the search for invariant properties of textual statistics \citeauthor{zipf1932selected} (\citeyear{zipf1932selected}). The existence
of such invariants suggested the possibility that some related
feature might be found that was at least invariant for any given
author, though possibly varying among different authors.
The majority of published works in authorship
attribution focus on closed-set attribution where
it is assumed that the author of the text under
investigation is necessarily a member of a given
well-defined set of candidate authors. This setting fits many forensic applications where usually specific individuals have access to certain resources, have knowledge of certain issues, etc.
A more general framework is open-set attribution \cite{koppel2011authorship}. A special case of the latter is authorship verification where the set of candidate authors is singleton. This is essentially a one-class classification problem since the negative class (i.e., all texts by all other authors) is huge and extremely heterogeneous.
In authorship attribution it is not always realistic to assume that the texts of known authorship and the texts under investigation belong in
the same genre and are in the same thematic area.
In most applications, there are certain restrictions
that do not allow the construction of a representative training corpus. Unlike other text categorization tasks, a recent trend in authorship attribution research is to build cross-genre and cross-topic models, meaning that the training and test
corpora do not share the same properties.
One crucial issue in any authorship attribution approach is to quantify the personal style of authors, a line of research also called stylometry \cite{stamatatos2009survey}. Ideally, stylometric features should not be affected by shifts in topic or
genre variations and they should only depend on
personal style of the authors. However, it is not
yet clear how the topic/genre factor can be separated from the personal writing style.

\paragraph{In this work} In this work we will give a general snapshot of what authorship attribution is nowadays. We will give a key of reading from the point of view of both information retrieval and the most used methodologies, paying much attention in the selection of datasets in order not to create learning bias.
In our experiment we will compare datasets that are very different from each other both in text content, formality and average length of documents written by authors.
We will analyze the methods to extract the best information from the documents and how to select the classifier that best suits this type of task.
We will compare the results obtained with those found in related work, taking care to recreate the same experiment performed.
Finally, we will give hints for future work and a long-term vision for this branch of text categorization.

\section{Thesis Structure}
The rest of this thesis is organized into the following chapters:
\begin{itemize}
	\item \textbf{Chapter 2}. Chapter 2 provides a more detailed introduction to this task and what are the different scenarios we may encounter when tackling the authorship attribution.
	\item \textbf{Chapter 3}. Chapter 3 presents a background in information retrieval, listing some of the most used techniques used in previous authorship attribution works for extracting useful data representing the texts.
	\item \textbf{Chapter 4}. Chapter 4 lists a variety of related works in authorship attribution, focusing on works that used SVM as a classifier and that used either one of the Reuters Corpus dataset or The Guardian dataset to test their model.
	\item \textbf{Chapter 5}. Chapter 5 describes our experiment in terms of the setup, the dataset preparation, the process of features extraction and the classifier selection.
	\item \textbf{Chapter 6}. Chapter 6 shows the best results we obtained from the experiments we conducted on the selected datasets.
	\item \textbf{Chapter 7}. Chapter 7 makes some observation on how can this work be improved in the next steps.
\end{itemize}