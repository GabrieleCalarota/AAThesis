\chapter{Result and Evaluation}
In this thesis we proposed domain-specific word embeddings for ICD-9-CM classification. In particular, we showed how to collect the dataset and enlarge it with other medical data.\\
We gathered together 3 main corpora of specific medical data: the first one from the emergency room discharge records of the Forlì Hospital, where we collected more than 700k real anonymous diagnosis written by doctors when sending home patients, the second one has been downloaded from all the italian medical articles available on Wikipedia and the last one was the official ICD-9-CM dictionary of the more than 16k definitions of diagnosis and their corresponding code.\\
For the general and comparison models, as we weren't able to use news data (i.e. Google-News, the most popular) because they aren't provided in italian language, we were forced to use other pre-trained models in italian. We found 2 of them with different backgrounds: the first one is the wikipedia dump of all the pages in it, the other one is provided with TED speech corpus translated in italian.\\
Due to the completely different nature of the two general purpose datasets, they showed when testing most similar words that they responded in a completely different way because they aren't filled with medical specific words in their starting dataset.\\
Subsequently, with the help of gensim tool and Word2Vec we trained each corpus separately forming 3 separate models and one joining all the documents of each corpus. Every model has been trained with the same hyperparameters.
We showed the results of the training of this dataset compared to the general purpose retrieved from TED corpus and general Wikipedia italian dump. Through most-similar words tool with the help of 2D plot, we tested the different models and showed how they respond.\\
We pointed out some of the most common feature typical in a medical text, such as typo errors and medical jargon. Medical terminology is made up of numerous Greek and/or Latin suffixes and prefixes. It describes body parts, functions, surgical procedures, and is used in medical reports. Every language has his own lingo and even every specific group of doctors has its own. When writing down an emergency room discharge records, doctors very often use abbreviations and sort of codes known specifically in the medical jargon to describe body parts and common words. This is done for plenty of reasons: first of all in Italy doctors still use quite a lot handwritten emergency room discharge records, but even typing on a keyboard doctors are known to be a little bit lazy.\\
From our evaluated models we looked for the meaning of main commons medical abbreviations and on the other hand we discovered how a word is commonly abbreviated by doctors.\\ 
The reader has to understand that this process is automatic and the model was unsupervised. Most similar words to a given word it is a result solely based on the context taken from the training dataset.\\
We finally proofed that our domain-specific word embeddings should be chosen over the general purpose. We presented a related work where we tried to automatically assign the codes corresponding to a textual diagnosis through a classifier. This classifier used our domain-specific word embedding for weighting words instead of using a general-purpose one and we noticed a real gain in terms of accuracy.\\
As we can clearly see from this last results, we can conclude that domain-specific word embeddings is more suitable for this task. Moreover the \enquote*{JoinData} model is undoubtedly the most accurate among the other word embeddings, mostly over the general purpose word embeddings. \\
As for what concerns the classification problem of ICD-9-CM it's still an hard task to solve, but we can unmistakably state that for word embedding of this type in a domain-specific context, such as the medical one filled with jargon, technique words and mostly typos, a domain-specific word embeddings performs better than a general purpose one.