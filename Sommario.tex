\sommario{%
  ICD (International Classification of Diseases), ovvero la classificazione internazionale delle malattie, è un sistema standard di classificazione ampiamente usato, che codifica un grande numero di specifiche malattie, sintomi, infortuni e procedure mediche in classi numeriche. Assegnare un codice ad un caso clinico significa classificarlo in una o più classi discrete, permettendo studi statistici e procedure di calcolo automatico. E' evidente che la possibilità di avere un codice discreto invece di una frase in linguaggio naturale ha un enorme vantaggio per i sistemi di manipolazione dei dati. L'uso di questo sistema di classificazione, ufficialmente alla decima revisione (ICD-10-CM)\footnote{Attualmente non ancora adottata in Italia}, diventa sempre più importante per ragioni legate alle polizze assicurative e potrebbe interessare anche i bilanci amministrativi dei reparti ospedalieri.\\
  Ottenere un classificatore automatico accurato è però un arduo compito in quanto la revisione ICD-9-CM conta più di 14 mila classi, quasi 68 mila nella revisione ICD-10-CM. Ottenere un training set soddisfacente per un Classificatore Testuale (TC) è quasi impossibile: sono rari i testi medici ufficiali etichettati con i codici, mentre le diagnosi reali da classificare sono scritte in gergo medico e piene di errori ortografici.\\
  Avendo un training set piuttosto ristretto ci aspettiamo che ampliandolo con un corpus di dati testuali in ambito medico, migliori l'accuratezza del classificatore automatico.\\
  Questo lavoro di tesi descrive innanzitutto come costruire e mettere insieme un dataset con soli dati testuali in ambito medico-specifico. Questo dataset viene, in secondo luogo, manipolato con la tecnica del \enquote*{word embedding} (i.e. \textit{immersione di parole}) che associa informazioni semantiche e sintattiche delle parole tramite 
  numeri, costruendo uno spazio vettoriale in cui i vettori delle parole sono più vicini se le parole occorrono negli stessi contesti linguistici, cioè se sono riconosciute come semanticamente più simili.\\
  Viene presentato, infine, come un word embedding di dominio specifico\footnote{In questo caso in ambito medico} aiuti a migliorare il classificatore automatico, e sia quindi preferibile ad un word embedding di tipo generico.
}