\sommario{%
  Il compito di determinare o verificare la paternità di un testo anonimo basandosi solo su analisi del documento è molto antico, e risale almeno all'era medievale, per la quale l'attribuzione affidabile di un testo ad un'antica autorità conosciuta era essenziale per determinarne la veridicità. Più recentemente, questo problema dell'attribuzione della \enquote{paternità} di un documento ha guadagnato maggiore importanza a causa delle nuove applicazioni nell'analisi forense, nelle scienze umanistiche e nel commercio elettronico, e lo sviluppo di metodi computazionali per affrontarlo.\\
  Nella forma più semplice, ci vengono dati esempi
  di scrittura di un certo numero di candidati autori e ci viene chiesto
  di determinare chi di loro è l'autore di un documento di autore ignoto. In questo caso, il problema dell'attribuzione dell'autore si adatta al paradigma moderno di un problema di categorizzazione del testo. I componenti dei sistemi di categorizzazione del testo sono ormai abbastanza ben compresi: I documenti sono rappresentati come vettori numerici che catturano le statistiche delle caratteristiche potenzialmente rilevanti del testo, e i metodi di apprendimento automatico sono utilizzati per trovare classificatori che separino i documenti che appartengono a classi diverse.\\
  La maggior parte dei lavori pubblicati si concentra sull'attribuzione a serie chiusa dove si presume che l'autore del testo preso in esame sia necessariamente un membro di un insieme ben definito di autori candidati. Questa impostazione si adatta a molte applicazioni forensi in cui di solito individui specifici hanno accesso a certe risorse, hanno conoscenza di certe questioni, ecc.
  In questo lavoro viene mostrata una panoramica generale di ciò che è l'attribuzione d'autore al giorno d'oggi, con uno sguardo sia dal punto di vista dell'information retrieval che delle metodologie più utilizzate, prestando molta attenzione nella selezione dei dataset per non creare bias di apprendimento.
  Vengono confrontati dataset molto diversi tra loro sia per contenuto testuale, livello di formalità del linguaggio e lunghezza media dei documenti scritti dagli autori.
  Confrontando il lavoro con altri correlati, verrà mostrato come alcuni risultati ottenuti sono potenzialmente interessanti e portino a migliorare ulteriormente la confidenza del modello per questo particolare tipo di task.
}